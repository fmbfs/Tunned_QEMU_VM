\documentclass[11pt, a4paper, oneside]{article}
\usepackage[english]{babel}	% Para caracteres em tuga
\usepackage[utf8]{inputenc} %para aceitar cedilhas
\usepackage{ulem}
\usepackage{adjustbox}
\usepackage{amsmath,amsthm,amsfonts,amssymb,amscd}
\usepackage{multirow,booktabs}
\usepackage[table]{xcolor}
\usepackage{xcolor}
\usepackage{ragged2e}
\usepackage{graphicx}
\usepackage{subfigure}
\usepackage{fullpage}
\usepackage{float}
\usepackage{lastpage}
\usepackage{enumitem}
\usepackage{fancyhdr}
\usepackage{hyperref}
\hypersetup{
    %bookmarks=true,    	     	 % show bookmarks bar?
    unicode=false,      	   		 % non-Latin characters in Acrobat's bookmarks
    pdftoolbar=true,    		     % show Acrobats toolbar?
    pdfmenubar=true,        	  	 % show Acrobats menu?
    pdffitwindow=true,     		  	 % window fit to page when opened
    pdfstartview={FitH},    	  	 % fits the width of the page to the window
    pdftitle={CTW},		 % title
    pdfauthor={Francisco Santos}% author
    pdfsubject={INTERNAL DOC},    % subject of the document
    pdfcreator={Francisco Santos},   				 	% creator of the document
    pdfproducer={}, 				 	% producer of the document
    pdfkeywords={},                  % list of keywords
    pdfnewwindow=true,  		     % links in new window
    colorlinks=true,    		     % false: boxed links; true: colored links
    linkcolor=black,      		     % color of internal links
    citecolor=black,    		     % color of links to bibliography
    filecolor=magenta,  		     % color of file links
    urlcolor=cyan        			 % color of external links
}
\usepackage{mathrsfs}
\usepackage{wrapfig}
\usepackage{setspace}
\usepackage{calc}


\usepackage{multicol} %colunas
\setlength{\columnsep}{1cm} %distancia entre colunas

\usepackage{cancel}
\usepackage{framed} %para meter as figuras e assim onde queremos
%\usepackage[retainorgcmds]{IEEEtrantools}
\usepackage[margin=3cm]{geometry}
\usepackage{amsmath}
\newlength{\tabcont}
\setlength{\parindent}{0.0in}
\setlength{\parskip}{0.05in}
\usepackage{empheq}
\usepackage[most]{tcolorbox}
\usepackage{xcolor}
%para ifenizar
\usepackage[none]{hyphenat}%%%%

\colorlet{shadecolor}{orange!15}
\parindent 0in
\parskip 12pt
\geometry{margin=1in, headsep=0.25in}

%adiconar notas
\theoremstyle{definition}
\newtheorem*{note}{Nota}

%Espacamento tabelas
\setlength{\tabcolsep}{5pt} % Default value: 6pt
\renewcommand{\arraystretch}{1.2} % Default value: 1

%data auto
\usepackage{datetime2}
\date{} %clear date

%remover as linhas
\renewcommand{\headrulewidth}{0pt}
\renewcommand{\footrulewidth}{0pt}

%=============================================================================================================
\begin{document}
\sloppy
\justifying

%para retirar numero de pagina
\pagenumbering{gobble}
\pagestyle{fancy}
\fancyhead{}
\fancyfoot[L]{\tiny{CTW}}

%titulo
\Large\textbf{CRITICAL TECHWORKS (CTW)}\\
\singlespacing
\Large\textbf{Notes for PESTA report}\\
\singlespacing
\indent\indent\indent\small{LICENCIATURA EM ENGENHARIA ELETROTÉCNICA E DE COMPUTADORES}\\
\singlespacing
\indent\indent\indent\small{INSTITUTO SUPERIOR DE ENGENHARIA DO PORTO}
\begin{spacing}{0.5}
\indent\indent\indent\small{POLITÉCNICO DO PORTO}
\end{spacing}
\singlespacing
\singlespacing
\singlespacing
\textbf{Lab Classes Script:}
\singlespacing
\indent\indent\indent\indent \textbf{Unity engine applied to Augmented Reality (AR)}


\vfill
\pagebreak
%---------------------------------------------------------------------------------------------------------------------------------
\section*{Version Control}
\begin{center}
\begin{tabular}{rl}
  \textbf{Professor:} & André Rocha\\
  \textbf{Class:} & 3NA\\
  \textbf{Group:} & 2
\end{tabular} 	
\singlespacing
\end{center}

\begin{table}[H]
	\centering
	\resizebox{.8\textwidth}{!}{
		\begin{tabular}{|>{\centering\arraybackslash}p{0.1\textwidth}>{\centering\arraybackslash}p{0.12\textwidth}>{\centering\arraybackslash}p{0.3\textwidth}>{\centering\arraybackslash}p{0.4\textwidth}|} 
			\hline
			\textbf{Version number} & \textbf{Date issued} & \textbf{Authors} & \textbf{Update information}\\
			\hline
		\end{tabular}
		}
		\resizebox{.8\textwidth}{!}{
		\begin{tabular}{ |>{\centering\arraybackslash}p{0.1\textwidth}|>{\centering\arraybackslash}p{0.12\textwidth}|>{\centering\arraybackslash}p{0.3\textwidth}|>{\centering\arraybackslash}p{0.4\textwidth}|}
			\textbf{V0} & 2022-03-10 & Francisco Santos & Original version of the script\\

		\end{tabular}
		}
\end{table}



\vfill
\pagebreak
%---------------------------------------------------------------------------------------------------------------------------------
%### TABELA DE CONTEUDOS ###

%mudar nome table of contents
\renewcommand*\contentsname{Table of Contents}

%tabela de conteudos
\tableofcontents


\fancyfoot[L]{\tiny{SISTCA}}
\vfill
\pagebreak
%---------------------------------------------------------------------------------------------------------------------------------

%volta a colocar paginas
\pagenumbering{arabic}

\fancyfoot[L]{\tiny{SISTCA}}
\fancyfoot[C]{}
\fancyfoot[R]{\small{\thepage$/$\pageref{LastPage}}}

%---------------------------------------------------------------------------------------------------------------------------------
%\begin{figure}[H]
%		\begin{center}
%			\includegraphics[width = 0.75\textwidth]{./figs/passo1.pdf}		
%			\caption{Choose Unity version to.}
%			\label{fig:passo1}
%		\end{center}
%\end{figure}




\vfill
\pagebreak
%================================================================================================================================
\end{document}
